%%%% Modèle proposé par kira.ribeiro@universite-paris-saclay.fr %%%%
%%%% màj : 29 octobre 2021 %%%%

\documentclass[french,12pt,a4paper]{book}
\usepackage[utf8]{inputenc}
\usepackage[T1]{fontenc}
\usepackage[french]{babel}
\usepackage[default,oldstyle,scale=.95]{opensans} % police Open Sans
\usepackage{amsmath}
\usepackage{amsfonts}
\usepackage{fancyhdr}
\usepackage{amssymb}
\usepackage{xcolor} % où color selon l'installation
\definecolor{Prune}{RGB}{99,0,60} % l14-33 : couleurs de la charte graphique upsaclay
\usepackage{mdframed}
\usepackage{multirow} %% Pour mettre un texte sur plusieurs rangées
\usepackage{multicol} %% Pour mettre un texte sur plusieurs colonnes
\usepackage{tikz}
\usepackage{graphicx}
\usepackage[absolute]{textpos} 
\usepackage{colortbl}
\usepackage{array}
\usepackage{geometry}
\usepackage{titlesec}
\usepackage{hyperref}
\hypersetup{ % paramétrage couleur des liens hypertextes, toujours garder colorlinks=true
    colorlinks=true,
    linkcolor=black,
    urlcolor=purple}


%\pagestyle{plain} % pour ne garder que les n°de page en milieu-bas et supprimer les indications de chapitre en marge haute

\usepackage{lipsum} % à retirer!!!

%*****************************************************
%******************** DOCUMENT **************************
%*****************************************************
\begin{document}

\begin{titlepage}

%\thispagestyle{empty}

\newgeometry{left=6cm,bottom=2cm, top=1cm, right=1cm}

\tikz[remember picture,overlay] \node[opacity=1,inner sep=0pt] at (-13mm,-135mm){\includegraphics{Frame-ups.pdf}};

%*****************************************************
%******** NUMÉRO D'ORDRE DE LA THÈSE À COMPLÉTER *****
%******** POUR LE SECOND DÉPOT                   *****
%*****************************************************

\color{white}

\begin{picture}(0,0)
\put(-152,-743){\rotatebox{90}{\Large \textsc{THÈSE DE DOCTORAT}}} \\
\put(-120,-743){\rotatebox{90}{NNT : 2020UPASA001}} % À MODIFIER
\end{picture}

%*****************************************************
%******************** TITRE **************************
%*****************************************************

\flushright
\vspace{10mm} % à régler éventuellement
\color{Prune}
\fontfamily{cmss}\fontseries{m}\fontsize{22}{26}\selectfont
  \Huge \textbf{Modelling self-assembly of \\ patchy virus rods \\}

\normalsize
\color{black}
\Large{\textit{Modélisation d'auto-assemblage de bâtonnets viraux}} \\
%*****************************************************

\fontfamily{fvs}\fontseries{m}\fontsize{8}{12}\selectfont

\vspace{1.5cm}

\normalsize
\textbf{Thèse de doctorat de l'université Paris-Saclay} \\

\vspace{6mm}

\small École doctorale n$^{\circ}$ 564, physique de l’Ile-de-France (PIF)\\
\small Spécialité de doctorat: Physique\\
\small Graduate School : GS Physique. Référent : Faculté des sciences d’Orsay \\
\vspace{6mm}

\footnotesize Thèse préparée dans l'unité de recherche \textbf{Université Paris-Saclay, CNRS, Laboratoire de Physique des Solides, 91405, Orsay, France}, sous la direction de \textbf{Rik WENSINK}, chargé de recherche CNRS. \\
\vspace{15mm}

\textbf{Thèse soutenue à Paris-Saclay, le JJ mois AAAA, par}\\
\bigskip
\Large {\color{Prune} \textbf{Marina TORRES LÁZARO}} % Changer le Prénom et le NOM

%************************************
\vspace{\fill} % ALIGNER LE TABLEAU EN BAS DE PAGE
%************************************

\bigskip

\flushleft
\small {\color{Prune} \textbf{Composition du jury}}\\
{\color{Prune} \scriptsize {Membres du jury avec voix délibérative}} \\
\vspace{2mm}
\scriptsize
\begin{tabular}{|p{7cm}l}
\arrayrulecolor{Prune}
\textbf{Prénom NOM} &   Président ou Présidente\\ 
Titre, Affiliation & \\
\textbf{Prénom NOM} &  Rapporteur \& Examinateur / trice \\ 
Titre, Affiliation   &   \\ 
\textbf{Prénom NOM} &  Rapporteur \& Examinateur / trice \\ 
Titre, Affiliation  &   \\ 
\textbf{Prénom NOM} &  Examinateur ou Examinatrice \\ 
Titre, Affiliation   &   \\ 
\textbf{Prénom NOM} &  Examinateur ou Examinatrice \\ 
Titre, Affiliation   &   \\ 
 

\end{tabular} 

\end{titlepage}


% page des résumés à garder en 2ème page. Si les résumés sont trop longs pour tenir sur une seule et même page, on peut mettre un résumé par page
\thispagestyle{empty}
\newgeometry{top=1.5cm, bottom=1.25cm, left=2cm, right=2cm}
\fontfamily{rm}\selectfont

\lhead{}
\rhead{}
\rfoot{}
\cfoot{}
\lfoot{}

\noindent 
%*****************************************************
%***** LOGO DE L'ED À CHANGER IMPÉRATIVEMENT *********
%*****************************************************
\includegraphics[height=2.45cm]{logo_usp_PIF.png}
\vspace{1cm}
%*****************************************************
\fontfamily{cmss}\fontseries{m}\selectfont

\small

\begin{mdframed}[linecolor=Prune,linewidth=1]

\textbf{Titre:} titre (en français)..................................................................................................................

\noindent \textbf{Mots clés:} 3 à 6 mots clefs (version en français)

\vspace{-.5cm}
\begin{multicols}{2}
\noindent \textbf{Résumé:}\lipsum[1-2] 
\end{multicols}

\end{mdframed}

\vspace{8mm}

\begin{mdframed}[linecolor=Prune,linewidth=1]

\textbf{Title:} titre (en anglais)..................................................................................................................

\noindent \textbf{Keywords:} 3 à 6 mots clefs (version en anglais)

\begin{multicols}{2}
\noindent \textbf{Abstract:} \lipsum[1-2]
\end{multicols}
\end{mdframed}

\titleformat{\chapter}[hang]{\bfseries\Large\color{Prune}}{\thechapter\ -}{.1ex}
{\vspace{0.1ex}
}
[\vspace{1ex}]
\titlespacing{\chapter}{0pc}{0ex}{0.5pc}

\titleformat{\section}[hang]{\bfseries\normalsize}{\thesection\ .}{0.5pt}
{\vspace{0.1ex}
}
[\vspace{0.1ex}]
\titlespacing{\section}{1.5pc}{4ex plus .1ex minus .2ex}{.8pc}

\titleformat{\subsection}[hang]{\bfseries\small}{\thesubsection\ .}{1pt}
{\vspace{0.1ex}
}
[\vspace{0.1ex}]
\titlespacing{\subsection}{3pc}{2ex plus .1ex minus .2ex}{.1pc}

\end{document}
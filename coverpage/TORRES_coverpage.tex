%%%% Modèle proposé par kira.ribeiro@universite-paris-saclay.fr %%%%
%%%% màj : 29 octobre 2021 %%%%

\documentclass[french,12pt,a4paper]{book}
\usepackage[utf8]{inputenc}
\usepackage[T1]{fontenc}
\usepackage[french]{babel}
\usepackage[default,oldstyle,scale=.95]{opensans} % police Open Sans
\usepackage{amsmath}
\usepackage{amsfonts}
\usepackage{fancyhdr}
\usepackage{amssymb}
\usepackage{xcolor} % où color selon l'installation
\definecolor{Prune}{RGB}{99,0,60} % l14-33 : couleurs de la charte graphique upsaclay
\usepackage{mdframed}
\usepackage{multirow} %% Pour mettre un texte sur plusieurs rangées
\usepackage{multicol} %% Pour mettre un texte sur plusieurs colonnes
\usepackage{tikz}
\usepackage{graphicx}
\usepackage[absolute]{textpos} 
\usepackage{colortbl}
\usepackage{array}
\usepackage{geometry}
\usepackage{titlesec}
\usepackage{hyperref}
\hypersetup{ % paramétrage couleur des liens hypertextes, toujours garder colorlinks=true
    colorlinks=true,
    linkcolor=black,
    urlcolor=purple}


%\pagestyle{plain} % pour ne garder que les n°de page en milieu-bas et supprimer les indications de chapitre en marge haute

\usepackage{lipsum} % à retirer!!!

%*****************************************************
%******************** DOCUMENT **************************
%*****************************************************
\begin{document}

\begin{titlepage}

%\thispagestyle{empty}

\newgeometry{left=6cm,bottom=2cm, top=1cm, right=1cm}

\tikz[remember picture,overlay] \node[opacity=1,inner sep=0pt] at (-13mm,-135mm){\includegraphics{Frame-ups.pdf}};

%*****************************************************
%******** NUMÉRO D'ORDRE DE LA THÈSE À COMPLÉTER *****
%******** POUR LE SECOND DÉPOT                   *****
%*****************************************************

\color{white}

\begin{picture}(0,0)
\put(-152,-743){\rotatebox{90}{\Large \textsc{THÈSE DE DOCTORAT}}} \\
\put(-120,-743){\rotatebox{90}{NNT : ************}} % À MODIFIER 2020UPASA001
\end{picture}

%*****************************************************
%******************** TITRE **************************
%*****************************************************

\flushright
\vspace{10mm} % à régler éventuellement
\color{Prune}
\fontfamily{cmss}\fontseries{m}\fontsize{22}{26}\selectfont
  \LARGE \textbf{Liquid crystal self-organization of colloids in complex environments}

\normalsize
\color{black}
\Large{\textit{Auto-organisation de cristaux liquides colloïdaux en milieux complexes}} \\
%*****************************************************

\fontfamily{fvs}\fontseries{m}\fontsize{8}{12}\selectfont

\vspace{1.5cm}

\normalsize
\textbf{Thèse de doctorat de l'université Paris-Saclay} \\

\vspace{6mm}

\small École doctorale n$^{\circ}$ 564, physique de l’Ile-de-France (PIF)\\
\small Spécialité de doctorat: Physique\\
\small Graduate School : GS Physique. Référent : Faculté des sciences d’Orsay \\
\vspace{6mm}

\footnotesize Thèse préparée dans l'unité de recherche \textbf{Université Paris-Saclay, CNRS, Laboratoire de Physique des Solides, 91405, Orsay, France}, sous la direction de \textbf{Rik WENSINK}, Chargé de recherche CNRS. \\
\vspace{15mm}

\textbf{Thèse soutenue à Paris-Saclay, le 30 juin 2023, par}\\
\bigskip
\Large {\color{Prune} \textbf{Marina TORRES LÁZARO}} % Changer le Prénom et le NOM

%************************************
\vspace{\fill} % ALIGNER LE TABLEAU EN BAS DE PAGE
%************************************

\bigskip

\flushleft
\small {\color{Prune} \textbf{Composition du jury}}\\
{\color{Prune} \scriptsize {Membres du jury avec voix délibérative}} \\
\vspace{2mm}
\scriptsize
\begin{tabular}{|p{7cm}l}
\arrayrulecolor{Prune}
%\textbf{Prénom NOM} &   Président ou Présidente\\
%Titre, Affiliation & \\
\textbf{Enrique VELASCO CARAVACA} &  Rapporteur \& Examinateur \\
Professeur, Autonomous University of Madrid  &   \\
\textbf{Hong XU} &  Rapporteur \& Examinatrice \\
Professeure, Université de Lorraine   &   \\
\textbf{Patrick DAVIDSON} &  Examinateur \\
Directeur de recherche, Université Paris-Saclay   &   \\
\textbf{Anja KUHNHOLD} &  Examinatrice \\
Chercheuse postdoctorale, University of Freiburg   &   \\
 

\end{tabular} 

\end{titlepage}


% page des résumés à garder en 2ème page. Si les résumés sont trop longs pour tenir sur une seule et même page, on peut mettre un résumé par page
\thispagestyle{empty}
\newgeometry{top=1.5cm, bottom=1.25cm, left=2cm, right=2cm}
\fontfamily{rm}\selectfont

\lhead{}
\rhead{}
\rfoot{}
\cfoot{}
\lfoot{}

\noindent 
%*****************************************************
%***** LOGO DE L'ED À CHANGER IMPÉRATIVEMENT *********
%*****************************************************
\includegraphics[height=2.45cm]{logo_usp_PIF.png}
\vspace{1cm}
%*****************************************************
\fontfamily{cmss}\fontseries{m}\selectfont

\small

\begin{mdframed}[linecolor=Prune,linewidth=1]

\textbf{Titre:} Auto-organisation de cristaux liquides colloïdaux en milieux complexes

\noindent \textbf{Mots clés:} cristal liquide, colloïde, théorie d'Onsager, Hard Particle Monte Carlo

% \vspace{-.5cm}
\begin{multicols}{2}
\noindent \textbf{Résumé:} Le travail présenté dans cette thèse se concentre sur une étude théorique de l'auto-organisation de cristaux liquides de bâtonnets ou de plaquettes colloïdales dans des milieux complexes. Les explorations sont faites à l'aide d'une théorie statistique basée sur la théorie d'Onsager et la théorie du champ moyen, combinées à des simulations numériques à grande échelle. Le premier cas que nous considérons est celui de disques colloïdaux immergés dans des bâtonnets réversiblement polymérisables où les deux composants sont capables de développer l'ordre nématique. Nous établissons les diagrammes de phase qui présentent un certain nombre de coexistences multiphasiques et discutons le phénomène de polymérisation réversible dans des environnements antinématiques. Le deuxième exemple concerne des bâtonnets et des plaquettes colloïdales immergés dans un cristal liquide thermotrope chiral. Il est démontré que ces cristaux liquides hybrides moléculaires-colloïdaux présentent un ordre biaxial amélioré, une séparation de phase liquide-liquide médiée par l'ancrage de surface et des organisations colloïdales-moléculaires hybrides bi-hélicoïdales exotiques à un contenu colloïdal significatif. Le dernier thème aborde l'auto-assemblage mésoscopique de bâtonnets colloïdaux chiraux mélangés à des polymères non absorbants. Selon les conditions, ces mélanges sont connus pour former des tactoïdes, des vaisseaux liquides en forme de membrane, ainsi que des rubans twistés.  En utilisant la simulation par ordinateur semi-grand canonique de Monte Carlo, nous abordons la morphologie et la structure interne de ces gouttelettes LC et nous comparons nos résultats avec des résultats expérimentaux récents sur des bâtonnets de virus fd filamenteux.
\end{multicols}

\end{mdframed}

\vspace{8mm}

\begin{mdframed}[linecolor=Prune,linewidth=1]

\textbf{Title:} Liquid crystal self-organization of colloids in complex environments

\noindent \textbf{Keywords:} liquid crystal, colloid, Onsager theory, Hard Particle Monte Carlo

\begin{multicols}{2}
\noindent \textbf{Abstract:} The work presented in this thesis focuses on a theoretical study of liquid crystal (LC) self-organization of colloidal rods or platelets in complex environments. Explorations are made using statistical theory based on Onsager and mean-field theory combined with large-scale computer simulations. The first case we consider are colloidal discs immersed in reversibly polymerizing rods where both components are capable of developing nematic order. We map out the phase diagrams that feature a number of multi-phase coexistences and discuss the phenomenon of reversible polymerization in anti-nematic environments. The second example concerns colloid rods and platelets immersed in a chiral thermotropic liquid crystal. These so-called hybrid molecular-colloidal liquid crystals are demonstrated to exhibit enhanced biaxial order, surface-anchoring mediated liquid-liquid phase separation, and exotic bi-helical hybrid colloidal-molecular organizations at significant colloid content. The last topic addresses the mesoscopic self-assembly of chiral colloidal rods mixed with non-adsorbing polymers. Depending on the conditions, these mixtures are known to form tactoids, membrane-shaped liquid rafts, as well as twisted ribbons.  Using semi-grand canonical Monte Carlo computer simulation, we address the morphology and internal structure of these LC droplets and compare our findings with recent experimental results on filamentous fd virus rods.
\end{multicols}
\end{mdframed}

\titleformat{\chapter}[hang]{\bfseries\Large\color{Prune}}{\thechapter\ -}{.1ex}
{\vspace{0.1ex}
}
[\vspace{1ex}]
\titlespacing{\chapter}{0pc}{0ex}{0.5pc}

\titleformat{\section}[hang]{\bfseries\normalsize}{\thesection\ .}{0.5pt}
{\vspace{0.1ex}
}
[\vspace{0.1ex}]
\titlespacing{\section}{1.5pc}{4ex plus .1ex minus .2ex}{.8pc}

\titleformat{\subsection}[hang]{\bfseries\small}{\thesubsection\ .}{1pt}
{\vspace{0.1ex}
}
[\vspace{0.1ex}]
\titlespacing{\subsection}{3pc}{2ex plus .1ex minus .2ex}{.1pc}

\end{document}
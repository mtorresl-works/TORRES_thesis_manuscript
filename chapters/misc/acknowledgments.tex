
\chapter{Acknowledgements}

I would like to begin by expressing my deepest gratitude to my supervisor, Rik Wensink, for all his support, guidance, and mentorship throughout this journey. His vast knowledge and constant feedback during our countless discussions have been invaluable in shaping this thesis. He also (and more importantly, I would say) has been a great inspiration to me on a human level: his humility, integrity, and patience have certainly influenced my way of doing science. His mentorship extended beyond research, and his genuine care for his students' well-being created a supportive and nurturing environment in which I was able to thrive both academically and personally even during the uncertain and stressful times of the pandemic.

Next, I extend my sincere appreciation to the French National Research Agency (ANR) for the financial support provided under Grant No. ANR-19-CE30-0024 “ViroLego”, as well as to the international research consortium INTEGRATE funded by the European Innovation Council. Their support has been instrumental in the successful completion of this work.

I acknowledge the esteemed jury members, Patrick Davidson, Enrique Velasco, Hong Xu, and Anja Kuhnhold, for their time, feedback, and valuable input during the evaluation of this thesis. I am also grateful to the members of my {\em comité de suivi}, Anniina Salonen and Patrick (again), for their constructive advice and insightful discussions throughout these years.

I would also like to thank the senior members of the Theo-Soft group, Giuseppe Foffi and Frank Smallenburg, whose expertise and interesting discussions during lunch have enriched my understanding of science and academia. I am also thankful to the post-doctoral researchers Andrea Plati, for his unalterable good vibes, and William Fall for trying to introduce me to LAMMPS and high-performance computing in a very short time, and of course for our fun times visiting {\em préfectures}. Gratitude is also extended to my office mates Saheli, Étienne, and Antoine, whose kind nature has cheered up my days. A special thanks to Étienne who invited me to move to his office when I was feeling lonely on the third floor during the pandemic. Additionally, I would like to acknowledge the valuable initial encouragement I received from a former Ph.D. student in our group, Susana.

I deeply appreciate the hospitality and encouragement I received from the other members of the Theo group. Special mention goes to the {\em other} Ph.D. students: Mateo, Baptiste, Ansgar, Jean-Baptiste, Lize and Florian with whom I repeatedly shared smiles and tears caused by this journey of tension and excitement in comparable proportions. Their friendship and discussions have been truly rewarding, and the good times we shared around a cup of coffee or a deck of cards will definitely be hard to forget.

I feel grateful to my friends from the undergraduate degree in Physics for sharing those early days when we still did not understand anything, and for staying in touch with me until today: Adri, Martín, Gabriel, Sara, Miguel, Julia, Marta, Álvaro, Isa. I could not have done it without you.

Being in Paris, I have had the opportunity to live with a strange group of fascinating people who are as curious about life in general as I am. I am talking about my friends from the Colegio de España in Paris, and they are too numerous to name them all. But I extend the message to whoever reads this text and knows that I am referring to them: there are no precise words that reflect how grateful I feel to have spent these years by your side, in the kitchen, at breakfast, studying late in the computer room, watching weird movies, sharing music and walls. Thank you a billion times over. And, in particular, thanks to Mateo (again) for offering me his support not only as a work colleague, but also as a neighbour, a playmate, a vegetable provider and a truly caring friend.

None of this would have been possible without the unconditional support of my family and close friends who, despite belonging to fields of expertise far removed from physics, have taken the time to listen to all my mad scientist ramblings. Thanks especially to my aunt Raquel, my aunt Inés, my cousin Clara and my friend Manuelita for coming from Madrid to be at my thesis defense, and to my friend Elba for being at my rehearsals via Zoom. Thanks to my aunt Carolina (to whom this thesis is dedicated) for being my French mother, despite not being French nor my mother; because of you this country always felt like home. And last but not least, Mamá, Papá, Elena, thank you for being the best team I could have ever asked for. Your love, encouragement, and belief in me have been my driving force since the day I was born (well, technically, since Elena was born). Your sacrifices and dedication inspired me to pursue everything I set my mind to, and I am eternally grateful for everything you have done for me. Os quiero muchísimo.

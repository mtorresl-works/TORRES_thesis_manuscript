
\chapter{General introduction}

\begin{abstract}
  \blue{
  In this Chapter we introduce the concept of lyotropic liquid crystals, both from a practical and statistical mechanical point of view.  We also establish the aim of this thesis in relation to recent experimental work on colloidal mixtures.
  }

\end{abstract}


\section{Phenomenological background}

{\em Soft matter} is a term used to describe materials that are distinct from gases and solids, usually excluding simple fluids. A wide range of systems, including soap bubbles, gels, elastomers, liquid crystals, or biological fluids, can be categorized as soft matter. Defining the boundaries of such a vast domain can be challenging due to the diverse topics and segregated nature of researchers, which come from a number of distinct disciplines. Nevertheless, a clear community with a common language exists, and major research topics can be identified that encompass most current studies in this field \cite{borsali2018soft}.

Soft matter systems exhibit, in most cases, structural length scales ranging from a nanometer up to a micrometer, and thus are placed within the domain of ‘nanotechnology’. Colloidal systems are well known examples in which this feature is essential to define the range where a very specific type of behavior occurs. Colloids are supramolecular submicron sized substances dispersed in a medium that can be a liquid or a gas. They are much bigger than normal molecules, and hence the medium in a colloidal suspension can often be regarded as ‘background’ with respect to the colloidal size range: this medium may be approximated as a continuum. At the same time, colloids are small enough to present considerable thermal motion in comparison to sedimentation (which is caused by gravitational forces that would become more important for higher-sized particles). Colloids were first discovered by Perrin, who detected Brownian motion as visible manifestation of thermal motion in dispersions of resin colloids in water \cite{perrin1913atomes}.

When colloidal particles have anisotropic shapes, they can present liquid crystalline phase behaviors. Liquid crystals are substances that have the appearance of a liquid but possess certain levels of molecular arrangement similar to crystals. Liquid crystals were first discovered in 1888 by Reinitzer, who noticed that a cholesterol-based substance had two melting points at different temperatures, each of them giving way to a liquid-like phase with different optical properties \cite{reinitzer1888beitrage}. At the early time of Reinitzer only three phases wew known (gas, liquid and solid). Over the years, a big number of substances have been discovered to exhibit many states of matter, including liquid crystals that are now widely used in technological advancements such as liquid crystal screens and thermometers \cite{Li_2012}.

The key difference between a liquid crystal and the commonly observed gas, liquid and solid states is that properties in the first one are anisotropic and vary with direction, even though the substance itself remains fluid. These unique properties emerge due to the elongated shape of its building blocks, which promote collective alignment along a certain direction. In other words, liquid crystalline phases are additional states of matter which are intermediate between the dilute gas and the crystalline solid, and whose existence is related to the additional {\em orientational} degrees of freedom anisometric particles have compared to spherical ones.

Among the numerous liquid crystalline phases, different degrees of order can be found, evidenced for instance by diffraction of X-rays and light. Measurements of this kind provide a frame to classify these systems by its similarity to either the gas or the solid phase. Let us consider, as an example, the following liquid crystalline phases:

The {\em isotropic} (I) fluid phase is very similar to the gas and liquid phases for spherical particles and is characterized by a complete absence of positional and orientational order. At the inmediately next stage, we find the {\em nematic} (N) phase, in which particles are homogeneously distributed without positional order as in a liquid phase, but are ordered in their orientation following an average direction: the {\em nematic director} $\bn$. As it will be repeatedly discussed throughout this thesis, in nature one can find chiral particles lacking of mirror symmetry. This is the case of helices or rods with chirally distributed charges at the surface, such as filamentous bacteriophages viruses named fd \cite{Gibaud_2017}. When these particles are in nematic phase, they arrange themselves into a strongly twisted structure. This special case of a nematic phase is often called {\em cholesteric}.

The {\em smectic} (Sm) phase is closer to the solid phase. In smectic liquid crystals, particles are ordered in layers and cannot move freely between them. The smectic phase is, in turn, divided into several sub-phases with slightly different properties. Examples are the smectic A phase (SmA), where particles can move freely inside the layers as in a two-dimensional liquid; or the smectic B phase (SmB), where there is long-ranged positional order: at higher concentrations or lower temperatures, molecules tend to arrange themselves into something more similar to a crystalline lattice.

One sub-classification of liquid crystal materials is based on the mechanism by which they transition from one state to another. Thermotropic systems, such as polymeric liquid crystals, undergo phase transitions due to changes in temperature -this is the case of molecular and polymeric liquid crystals-, while lyotropic liquid crystals form upon increasing the concentration of solute particles. In this thesis, we focus on lyotropic liquid crystals, specifically those composed of viral fd rods as the fundamental building block. Filamentous viruses are a well-studied model system for elongated colloidal particles due to their regular shape and size, high abundance, and the ability to genetically modify and chemically modify their surface.

Our research is focused on both the theoretical framework and computer simulations of phase behavior in anisotropic colloidal compounds, with a particular interest in the isotropic-nematic phase transition. This transition is well-described by the Onsager theory, which was proposed in 1949 and assumes a similarity between a gas and a particle solution \cite{onsager1949}. Through our research, we hope to gain a better understanding of the behavior of lyotropic liquid crystals composed of fd viruses and contribute to the broader field of liquid crystal research.

\blue{

\section{Phenomenological background}


It is  well-known from everyday experience that, under certain conditions, matter can transform from one state to the other
by a {\em phase transition}. For instance, increasing the temperature may lead to
boiling and evaporation of water or the melting of ice, nitrogen gas can be liquefied
at low temperatures and high pressures.
As to the liquid crystals we discuss here,  phase transitions among the different
states may be brought about
in two different ways; one by varying the temperature  and the other by changing
the concentration of particles in solution. The variety of systems characterized by the former
is qualified as `thermotropic' and consists mainly of systems of anisometric low molecular weight constituents (and also certain polymers).
In this thesis we shall however restrict ourselves to the latter class of systems; the {\em lyotropic}
liquid crystals. These are composed of high-molecular weight colloidal particles, polymers
or surfactants in a solvent such that the formation of these liquid crystals occurs upon increasing the {\em concentration} of the solute particles.

Historically, lyotropic liquid crystals were first recognised in the 1920s by Zocher \cite{Zocher}
who investigated nematic textures in solutions of rodlike inorganic vanadiumpentoxide (V$_{2}$O$_{5}$) particles.
Later, similar observations were reported by Langmuir \cite{Langmuir} for clay platelets
and Bawden {\em et al.} \cite{Bawden,Bernal} for Tobacco Mosaic Virus (TMV) rods.
At present, there are many other examples of lyotropic liquid crystals to be found in a wide
variety of dispersions of (mainly rodlike) colloidal particles and solutions of stiff polymers
(see e.g. Refs. \cite{Vroege92,Gabrieloverzicht} for an overview).
In the last decade much of the experimental effort in colloid science
was focussed on the development and characterization of colloidal {\em model systems} comprising particles
with a well-defined size and shape. The effective interactions between the particles can often be tailored
either by chemically altering the surface of the particles or by changing the solvent
conditions through variation of the ionic strength or the addition of non-adsorbing polymer.
As to anisometric colloids, two  important examples of these model systems are the  Boehmite  (AlOOH)  rods
\cite{Buiningwater,vanBruggen}
and the plate-shaped  Gibbsite (Al(OH$)_{3}$) particles \cite{Wierenga}.
These particles can be stabilised for instance by grafting a layer of polymer onto the particle surface
 \cite{vanBruggenPIB,vanderKooij}. If the particles are subsequently dispersed in a suitable solvent,
the polymer layer acts as a steric stabilizer which gives rise to short-ranged repulsive interactions,
closely resembling so-called {\em hard} interactions, i.e. the particles repel each other when they touch (they are
impenetrable) but do not interact otherwise.
Owing to their simple hard-body interactions the sterically stabilized systems
are particularly suitable for studying the influence of {\em particle shape}, explored by either
changing the intrinsic shape of the particles or by mixing particles with distinctly different sizes and shapes,
on the liquid crystalline phase behaviour of anisometric colloids \cite{felixthesis}.

\subsection{Entropic phase transitions}

On the theoretical side, the field of statistical mechanics of lyotropic
liquid crystals  was opened up by Lars Onsager in the 1940s. He recognized that the transition
from an isotropic to a nematic state in solutions containing sufficiently anisometric particles
can be described successfully  within a virial expansion of the free energy truncated after
the second virial term, an approach which could not be used to explain the gas-liquid transition for spherical particles.
 One of the crucial insights of Onsager was that the transition
can be explained by hard-body repulsions only, thereby dismissing the widespread notion
that attractive forces between particles  must be responsible for the formation of
 aligned configurations. The theory of Onsager will be described comprehensively in  Sec. 1.3.


About a decade after Onsager's work, Alder, Wainwright and others \cite{ALDER57,WOOD57} first showed by means of computer simulations that
a similar disorder-order transition, albeit of the positional degrees of freedom, occurs in a fluid of hard spheres. Subsequent work by
Hoover and Ree \cite{HooverRee} established that if the hard-sphere packing fraction
exceeds $\phi=0.494$ a fluid  spontaneously freezes into a crystalline
solid phase with a packing fraction $\phi=0.545$.
Much later, computer simulations by Frenkel {\em et al.}  revealed the stability
of smectic and columnar liquid crystals which appear upon densifying systems of respectively hard rods \cite{Frenkel88} and
 hard platelets \cite{FrenkelLiqcryst,Veerman},  without  attractive interactions between the particles.
The transitions of matter mentioned here share an important characteristic; they are driven purely by {\em entropy}, i.e. there are no energetic effects involved.
For this reason these ordering phenomena are nowadays
often referred to as {\em entropic phase transitions} \cite{Frenkelsoft}.
The notion that spontaneous ordering
of particles corresponds to an increase of the total entropy  may seem counter-intuitive
at first sight, since an increase  of order is usually connected to a decrease of entropy.
Yet, the general mechanism behind these transitions can be understood as follows.
Although the particles lose entropy because the density --in terms of orientations or positions--
is no longer uniform, this loss is more than offset by the simultaneous gain of translational
entropy, i.e.  the available space per particle increases as the particles align or freeze into
a crystal lattice.

\subsection{Mixtures}

So far we have implicitly assumed that all particles which build up a gas, liquid (crystal) or solid phase
are identical. Many systems in nature are however {\em mixtures} containing a number of different types of
particles or molecules. In this respect we may roughly distinguish
between mixtures of chemically distinct moieties on the one hand and so-called {\em polydisperse} mixtures on
the other. Examples of the first are blood
(containing red and white blood cells, plasma etcetera),
mayonnaise (a mixture of oil and vinegar) and milk (consisting of dispersed fat globules, casein micelles and
whey proteins).
Polydisperse mixtures are characterized by a large (potentially infinite) number of
species --all belonging to a {\em single} family of particles-- with  continuously varying properties such as particle shape, size or possible surface charge.
Common examples of these are colloidal model systems , where the particles usually have the same
basic shape (e.g.  spheres, rods or plates) but  a {\em range} of
radii, lengths, diameters, etcetera. Of course, in reality many systems
share characteristics of both classes; they
may comprise a number of distinct species  each with some degree of polydispersity with respect to
one or more properties of the particle family.

It is not surprising that the phase behaviour of  mixtures is richer than that of pure systems
--if only for the additional {\em entropy of mixing}-- and
that mixing different species may lead to  phenomena not encountered in one-component systems.
In particular, depending on the miscibility of the species involved,  mixing
may sometimes lead to a destabilization of a homogeneous state
causing a phase separation into two or more phases, each with a different
density and/or composition. The associated segregation of species among the coexisting phases,
called  {\em fractionation}, is inherent to mixtures and may sometimes give rise to surprising phenomena,
as shown in the next Chapter.
Mixing particles with distinctly different sizes or shapes
may also cause the formation of ``new" phases whose structures are not observed in pure systems
of the constituent species. For instance, in a two-component (or binary) mixture
 of rod- and platelike colloids we may encounter the so-called {\em biaxial} nematic
phase, in which both particles types are aligned
in mutually perpendicular directions. Similarly, mixing spheres of two (or more) different sizes
may give rise to the formation of various solid states with intricate
lattices structure  not encountered in pure solids \cite{Bartlettbinary}.


\section{Scope of this thesis}
The central aim of this thesis is to theoretically investigate
the effects of  {\em mixing}  particles with different shapes
on their liquid crystal phase behaviour.
Many of the studies to be described in the remainder of this thesis have been triggered off
by recent experimental observations in mixtures of colloids with well-controlled shapes
and interactions.
In particular, we mention the experimental work of Van der Kooij \cite{felixthesis} who investigated a vast number
of mixtures which display many interesting phenomena left open for theoretical interpretation.
One of our primary goals in this work is to account for these
experimental observations by constructing simple, yet realistic,  models
for the colloidal systems under consideration
and by scrutinizing relevant aspects of their phase behaviour.

The first part of this thesis will be devoted to binary mixtures of anisometric particles.
In Chapter 2 a simple model is proposed that allows to qualitatively explain
the recently observed isotropic-nematic density inversion in polydisperse systems of colloidal platelets.
In the next two chapters  we shall be concerned with mixtures of rods and platelets and
provide a theoretical underpinning for the low-concentration part of the experimental
phase diagram. We also assess the possible stability of the disputed biaxial nematic phase
in experimentally realizable mixtures.
In Chapter 5 we conclude the first part with  an overview on demixing transitions
within the isotropic and nematic phases of binary mixtures of particles
whose size differs only in one particle dimension. Previously published results for rodlike particles
will be combined with new results for platelets to compare phase diagram topologies and
demixing mechanisms pertaining to the various mixtures.

In the second part of this thesis we address the more challenging issue of calculating
phase equilibria in polydisperse mixtures of anisometric particles.
In Chapter 6 we present a study of isotropic-nematic phase coexistence in systems of
length-polydisperse hard rods, focussing in particular on fractionation effects and
the possibility of a demixing of the nematic phase.
Chapter 7 deals with polydisperse systems of thickness-polydisperse
platelets. The binary model, introduced in Chapter 2, is extended to a polydisperse
one which allows us to provide a more realistic, albeit still qualitative, description of the experimental observations.
In Chapter 8 we provide a preliminary
calculation on the competition
between smectic and columnar ordering in systems of polydisperse hard rods.  As a first-order
approximation we consider an artificial model system of {\em perfectly aligned} cylinders. The  possibilities of extending the approach towards a more realistic one will be discussed.


The contents of  Chapter 9 of this thesis differ somewhat from the  rest because of the introduction of an  {\em external field}.
Inspired by recent experimental observations of a significant sedimentation in dispersions of platelets
 we illustrate the drastic effect of gravity
 on the phase behaviour of colloidal mixtures. As an example we consider a system
of sedimenting platelets mixed with non-sedimenting ideal polymers, as studied
experimentally by Van der Kooij. Also here,  the results of the calculations
reveal an improved description of the experimentally observed
behaviour.

Finally, in Chapter  10  we present a free-volume theory for a columnar state of hard platelets  by combining the traditional cell model with an appropriate fluid description which accounts for the rotational freedom of the particles in the (one-dimensional) direction of the phase. Excellent quantitative agreement is found with recent computer simulation results.



\section{Statistical mechanical background}
\noindent In this section we introduce the statistical mechanical framework
of  Onsager's second virial theory \cite{onsager1949,Vroege92,Cotter} to
describe the thermodynamic properties of a spatially homogeneous fluid of
hard colloidal rods or
platelets. The theory is then modified to account for
higher virial terms by means of a decoupling approximation as devised by Parsons \cite{Parsons}.
Finally, we introduce a bifurcation analysis to verify  the stability of the  fluid
with respect to the spatially {\em inhomogeneous} liquid crystalline states.

\subsection{Fluids of hard anisometric particles}
We start from an (imperfect) gas of $N$ identical cylindrically symmetric particles in a volume $V$.
Assuming a pairwise additive interaction potential we can express the total potential energy $U_{N}$  as a summation over pairs
\begin{equation}
U_{N}=\sum_{i<j} u(\bfr_{ij}; \Omega_{i}, \Omega_{j}),
\end{equation}
where $\bfr_{ij}=\bfr _{j}- \bfr _{i}$ is the vector connecting the centres of mass of particles $i$ and $j$;
$\Omega_{i}$ and $\Omega_{j}$ represent the solid angles describing the orientations of
the respective particles with respect to some space-fixed coordinate system.
For hard-core interactions the pair potential explicitly reads
\begin{equation}
u(\bfr_{ij}; \Omega_{i}, \Omega_{j}) =
\begin{cases}
\infty  &\text{if $i$ and $j$ overlap;} \\
0 &\text{otherwise.} \label{0pairpot}
\end{cases}
\end{equation}
The  description can be applied analogously to dispersions of colloidal particles
 but the direct pair potential $u$
 should then be replaced by the potential of mean force $w(\bfr_{ij};\Omega_{i},\Omega_{j})$ describing
the interaction between the  particles $i$ and $j$
dispersed in a  solvent with a fixed chemical potential \cite{macmillan,Hill}.
This procedure involves a configurational average of the solvent molecules accounting for
their mutual interactions and the interactions with the dispersed particles.

The configurational partition function $Q_{N}$ for the system reads
\begin{equation}
Q_{N}=\frac{1}{\mathcal{V} ^{N} N!}\int d\bfr^{N}\int d\Omega ^{N}\exp\left[-\beta U_{N}(\bfr^{N}; \Omega^{N})\right],
\label{0qdef}
\end{equation}
with $\beta= 1/k_{B}T$ and $\mathcal{V}$ the (de Broglie) thermal volume, arising from
integrations over the translational and rotational momenta of the anisometric particles.
The positional and orientational degrees of freedom of the particles are collectively denoted by
$\{\bfr^{N};\Omega^{N}\}$.
If we assume that there are no  {\em a priori} restrictions on the particle orientations
we may approximate the  angular integrations  to arbitrary accuracy by dividing
the orientational phase space, i.e. the surface of a unit sphere, into $s$
arbitrarily small equal sections  with a surface $\Delta \Omega = 4\pi/s$ and summing over all
possible orientation distributions $\{N_{1},N_{2}, \ldots, N_{s}\}$, where $N_{k}$ is the
number of particles with its solid angle $\Omega$ in the $k$-th section centered about
$\Omega_{k}$ such that
\begin{equation}
\sum _{k=1}^{s} N_{k} =N. \label{0behoud}
\end{equation}
The partition function Eq. (\ref{0qdef}) then becomes
\begin{align}
Q_{N}=&\frac{1}{\mathcal{V} ^{N} N!}\left(\frac{\Delta \Omega}{4\pi}\right)^{N}
\sum_{N_{1}=0}^{N}\ldots \sum_{N_{s}=0}^{N}\frac{N!}{\prod_{k=1}^{s}N_{k}!} \nonumber \\
&\times\int d\bfr^{N}
\exp\left[-\beta U_{N}(\bfr^{N}; N_{1},N_{2},\ldots,N_{s})\right], \label{0Qalgemeen}
\end{align}
where the summations need to be carried out under the condition of Eq. (\ref{0behoud}).
For large $N$ it is justified to replace the sum  by its maximum term \cite{Hill}.
Denoting the set (i.e. orientation distribution) which maximizes $\ln Q_{N}$ (and hence $Q_{N}$) by
 $\{\tilde{N}_{1},\tilde{N}_{2},\ldots,\tilde{N}_{s}\}$ we obtain
\begin{equation}
Q_{N}\approx \frac{1}{\mathcal{V} ^{N} N!}\left(\frac{\Delta \Omega}{4\pi}\right)^{N}
\frac{N!}{\prod_{k=1}^{s}\tilde{N}_{k}!}\int d\bfr^{N}
\exp\left[-\beta U_{N}(\bfr^{N}; \tilde{N}_{1},\tilde{N}_{2},\ldots,\tilde{N}_{s})\right].
\end{equation}
The partition function can be expressed in a more convenient form after some rearranging. This yields
\begin{equation}
Q_{N}= \underbrace{\frac{V^{N}}{\mathcal{V} ^{N} N!}}_{Q_{N}^{\text{trans}}}
\underbrace{\left(\frac{\Delta \Omega}{4\pi}\right)^{N}
\frac{N!}{\prod_{k=1}^{s}\tilde{N}_{k}!}}_{Q_{N}^{\text{orient}}}
\underbrace{\left < \exp\left[-\sum_{i<j}\beta w(\bfr_{ij};
\tilde{N}_{i},\tilde{N}_{j})\right] \right >_{\{\tilde{N}_{1},\tilde{N}_{2},\ldots,\tilde{N}_{s}\}}
}_{Q_{N}^{\text{int}}},
\end{equation}
where the brackets denote a (normalized) configurational average over all positional and orientational
coordinates under the condition that the particles obey an orientational distribution
according to the set $\{\tilde{N}_{1},\tilde{N}_{2},\ldots,\tilde{N}_{s}\}$.
The first terms $Q_{N}^{\text{trans}}$ and $Q_{N}^{\text{orient}}$ are  identified as the translational (ideal gas) and
orientational contributions, respectively, whereas the bracketed one accounts for the hard-body interactions between
the particles.
The Helmholtz free energy is obtained from the standard relation $\beta F= -\ln Q_{N}$.
Applying this to $Q_{N}^{\text{trans}}$ gives the common ideal free energy
$\beta F_{\text{id}}= N\left[\ln (\rho \mathcal{V})-1\right]$, with $\rho=N/V$ the number density.
For the orientational part we obtain
\begin{equation}
\beta F_{\text{orient}}=N\left\{\ln \left[\frac{4\pi}{\Delta \Omega}\right] + \sum_{k=1}^{s}\tilde{n}_{k}\ln \tilde{n}_{k} \right\},
\label{0fordiscreet}
\end{equation}
in terms of the number fractions $\tilde{n}_{k}=\tilde{N}_{k}/N$ with $\sum_{k=1}^{s}\tilde{n}_{k}=1$.
Introducing the normalized {\em orientational distribution function} (ODF) $f(\Omega_{k})$ we may
write $\tilde{n}_{k}=f(\Omega_{k})\Delta\Omega$. Using this in Eq. (\ref{0fordiscreet}) and
taking the  limit $\Delta \Omega \rightarrow 0$ for a {\em continuous} distribution in $\Omega$ we
obtain the following expression for the orientational free  energy
\begin{equation}
\frac{\beta F_{\text{orient}}}{N}=\int f(\Omega)\ln \left[4\pi f(\Omega)\right]d\Omega.
\label{0forient}
\end{equation}
The configurational partition function  $Q_{N}^{\text{int}}$ can be approximated
systematically by a {\em virial expansion} in terms of the density variable $\rho$ \cite{hansenmacdonald}.
At low densities it is justified to make a second virial approximation   by taking each of the $N(N-1)/2$ pair interactions independent from all others
so that
\begin{align}
Q_{N}^{\text{int}}&= \left \langle \prod _{i<j}
\exp\left[-\beta w(\bfr _{ij}; \Omega_{i}, \Omega_{j})\right]  \right \rangle _{f}
\approx \prod _{i<j}
\left < \exp\left[-\beta w(\bfr _{ij}; \Omega_{i}, \Omega_{j})\right]  \right>_{f} \nonumber \\
&\approx\left<1+\Phi_{12}\right>_{f} ^{N(N-1)/2}. \label{0secondvir}
\end{align}
The subscript $f$ indicates the condition that the orientational distribution is given by
$f(\Omega)$. Furthermore, $\Phi_{12}$ is the Mayer function, defined as
\begin{equation}
\Phi_{12}\equiv \exp\left[-\beta w(\bfr _{12}; \Omega_{1}, \Omega_{2})\right]-1.
\end{equation}
Applying the hard-core pair potential Eq. (\ref{0pairpot}) we see that this function is equal to -1  if
two particles $1$ and $2$ overlap and zero otherwise.
Spatially integrating the Mayer function
yields the so-called pair cluster integral $\beta _{1}$:
\begin{equation}
\beta_{1}(\Omega_{1},\Omega_{2})\equiv \frac{1}{V} \int d\bfr_{1}d \bfr_{2} \Phi(\bfr _{12}; \Omega_{1}, \Omega_{2})=-v_{\text{excl}}(\Omega_{1},\Omega_{2}),
\label{0cluster}
\end{equation}
which is equal to minus the {\em excluded volume} $v_{\text{excl}}$ of two anisometric particles at fixed
solid angles $\Omega_{1}$ and $\Omega_{2}$.
Using this in Eq. (\ref{0secondvir}) yields
\begin{align}
Q_{N}^{\text{int}}&\approx \left[1-\frac{1}{V} \iint d\Omega_{1} d\Omega_{2}
f(\Omega_{1})f(\Omega_{2})v_{\text{excl}}(\Omega_{1},\Omega_{2})\right]^{N(N-1)/2} \nonumber \\
&\approx \exp\left[-N \frac{\rho}{2}  \iint d\Omega_{1} d\Omega_{2}
f(\Omega_{1})f(\Omega_{2})v_{\text{excl}}(\Omega_{1},\Omega_{2})  \right].
\end{align}
Collecting results we obtain the following expression for the free energy
of a fluid of hard anisometric particles in the second virial approximation:
\begin{align}
\frac{\beta F}{N} =& \beta \mu_{0}+\ln\left[\mathcal{V}\rho\right]-1+
\int f(\Omega)\ln \left[4\pi f(\Omega)\right]d\Omega \nonumber \\
&+\frac{\rho}{2} \iint d\Omega d\Omega^{\prime}
f(\Omega)f(\Omega^{\prime})v_{\text{excl}}(\Omega,\Omega^{\prime}). \label{0freetot}
\end{align}
with $\mu_{0}$ a reference chemical potential of the dispersed particles depending
only on the solvent conditions. Higher order contributions in the virial expansion of the free energy
--involving clusters of three, four, etcetera particles--
can be derived using similar arguments as in Eq. (\ref{0secondvir}) \cite{vankampen}.
In the {\em third virial} approximation for example we encounter the triplet cluster
integral $\beta_{2}(\Omega_{1},\Omega_{2},\Omega_{3})$
\begin{equation}
\beta_{2}(\Omega_{1},\Omega_{2},\Omega_{3})\equiv \frac{2}{V}\iiint
d\bfr_{1}d\bfr_{2}d\bfr_{3} \Phi_{12}\Phi_{13}\Phi_{23}, \label{0beta2}
\end{equation}
which is nonzero only if three particles overlap simultaneously.
Eq. (\ref{0beta2}) and higher order cluster integrals are notoriously difficult to calculate
because this requires knowledge of the excluded volume of a multi-particle
cluster as a function of the orientations of all particles involved.
In practice, other methods are adopted to include higher virial terms, albeit
approximately, such as
`scaled particle' \cite{Cotterspt,Cotter} and density functional theories (see \cite{Vroege92,DFTspecialJPCM} for a review).
In this thesis we shall often use the so-called decoupling approximation, to be described
in Sec. 1.3.3.


The next step is to minimize the free energy, at a given density $\rho$, with respect to
the non-conserved orientational degrees of freedom.
In practice, there are two different ways to find this minimum; a formal approach and
a trial function method which we both shall discuss briefly here.
 The formal way  is to apply a functional differentiation of the free energy
with respect to the ODF $f(\Omega)$. This yields the stationarity condition:
\begin{equation}
\frac{\delta}{\delta f(\Omega)} \left[\frac{\beta F}{N}-\lambda^{\prime} \int f(\Omega)d \Omega \right]=0,
\label{0statcond}
\end{equation}
where $\lambda^{\prime}$ is a Lagrange multiplier to be determined from the
normalization condition for the ODF:
\begin{equation}
\int f(\Omega)d\Omega =1.
\end{equation}
Inserting the free energy Eq. (\ref{0freetot}) gives a nonlinear integral equation
\begin{equation}
\ln[4\pi f(\Omega)]=\lambda - \rho \int f(\Omega^{\prime})v_{\text{excl}}(\Omega,\Omega^{\prime})
d\Omega^{\prime}, \label{0inteq}
\end{equation}
with $\lambda=\lambda^{\prime}-1$. A trivial solution to Eq. (\ref{0inteq}) is
the constant $f(\Omega)=1/4\pi$ describing an {\em isotropic} fluid in which
all particle orientations are equally probable.
The thermodynamic equilibrium ODF for a {\em nematic} state --which
will be a peaked function--
can however only be obtained
 numerically e.g. in terms of a series expansion in
Legendre polynomials \cite{kayser,lasher,lakatos} or by means of a discretization scheme \cite{herzfeldgrid}.
It is important to realize that for  {\em uniaxial} particles the ODF satisfies
both azimuthal symmetry around the nematic director and inversion symmetry.
The former implies that the ODF depends only on the  polar angle $\theta$
between the particle orientation vector and the nematic director\footnote{An exception to this case is a mixture of uniaxial rods and plates,
where the ODF may depend on the azimuthal angle as well
due to a possible {\em biaxial} symmetry of the nematic phase.
This will become clear in Chapter 4.}, so that $f(\Omega)=f(\theta)$.
The latter  implies the angles $\theta$ and $\pi-\theta$ being equivalent, thus $f(\theta)=f(\pi-\theta)$.



To avoid the necessity of solving the nonlinear integral equation Eq. (\ref{0inteq}) one may
choose the trial function approach instead. The ODF $f(\Omega)$ in Eq. (\ref{0freetot}) is then
replaced by a {\em fixed functional form}
depending on one or more variational parameters and the free energy is subsequently minimized
with respect to these parameters.
This approach was first employed by Onsager  in his original paper \cite{onsager1949} where he used the following
trial form:
\begin{equation}
f_{O}(\cos \theta) =\frac{\alpha \cosh (\alpha \cos\theta)}{4\pi\sinh \alpha},
\end{equation}
in terms of the variational parameter $\alpha$.
Although this trial function gives reasonable results for the isotropic-nematic
phase transition the analysis involved is
quite complicated.
In this thesis we shall therefore often use the simpler Gaussian trial ODF, introduced
by Odijk \cite{OdijkLekkerkerker}:
\begin{equation}
f_{G}(\theta ) \cong   \left\{
\begin{tabular}{lll}
$Z\exp [-\frac{1}{2}\alpha\theta ^{2}]$ & if & $%
0\leq \theta \leq \frac{\pi }{2}$ \\
&  &  \\
$Z\exp [-\frac{1}{2}\alpha (\pi -\theta )^{2}]$
& if & $\frac{\pi}{2}\leq \theta \leq \pi $%
\end{tabular}
\right.  \label{0ODF}
\end{equation}
The normalization constant $Z=Z(\alpha)$ can be calculated analytically
by means of an asymptotic expansion for large $\alpha$. Noting
that $f_{G}(\theta)$ is then a rapidly decaying function we may expand as follows
\begin{align}
Z&=\left[\int_{0}^{2\pi} d \phi \int _{0}^{\pi} \exp \left[-\frac{1}{2}\alpha\theta ^{2}\right] \sin\theta d\theta\right]^{-1}
\sim \left[ 4\pi \int _{0}^{\infty} \exp \left[-\frac{1}{2}\alpha\theta ^{2}\right]\left\{ \theta - \frac{1}{6} \theta ^{3}+\ldots \right \}
 d\theta  \right]^{-1} \nonumber \\
&\sim \frac{\alpha}{4\pi}\left(1+\frac{1}{3\alpha}+\ldots \right), \label{0norma}
\end{align}
where the error introduced by extending  the $\theta$-integration  to infinity is $\mathcal{O}
(\text{e}^{-\alpha})$. Retaining the leading order term in Eq. (\ref{0norma}) we see that
the Gaussian ODF, unlike $f_{O}(\cos\theta)$,
does {\em not} give the correct isotropic ODF $1/4\pi$ in the limit $\alpha \rightarrow 0$, since $f_{G}(\theta)$
vanishes in this limit. Using the Gaussian ODFs
we can calculate the {\em typical} or root-mean-square polar angle, which is related to the variational parameter
via $\langle\theta^{2}\rangle^{1/2}\propto  \alpha^{-1/2}$, showing that it
will be small for large $\alpha$. We shall  use this relation implicitly in Chapter 5
where  an alternative description of the trial function approximation will be presented
 entirely in terms of these typical angles.


A benefit of using the Gaussian trial ODF
is that it renders the Onsager theory analytically tractable.
The free energy minimization can be carried out entirely analytically,
which reveals that $\alpha\propto \rho^{2}$,
whereas approximate asymptotic expressions
can be derived for the orientational and
excess parts of the free energy Eq. (\ref{0freetot}) \cite{Vroege92,OdijkLekkerkerker}.
Although the Gaussian ODF
is {\em not} a solution of the the exact stationarity condition Eq. (\ref{0statcond})
it  does satisfy an exact high-density scaling relation for the
ODF, as was shown by Van Roij \cite{vanroijmulderscaling},
 owing to the abovementioned
quadratic density-dependence of the variational parameter.
This in turn implies that the Gaussian ODF is particularly suitable
for strongly ordered nematic phases.
As to  mixtures of rods with different lengths, the Gaussian ODF
has so far been successful in explaining the generic features of the isotropic-nematic
phase behaviour
such as a fractionation effect, a widening of biphasic gap \cite{OdijkLekkerkerker} and the
existence of triphasic and nematic-nematic equilibria \cite{LekkerVroeg}.





%In this section we present an analytical theory based on the approximate
%Gaussian trial orientation distribution function (ODF) as formulated by
%Odijk {\em et al.} \cite{OdijkLekkerkerker}, which is a simplified version of the
%trial ODF used by Onsager \cite{Onsager}.


%In this section we present an analytical theory based on the approximate
%Gaussian trial orientation distribution function (ODF) as formulated by
%Odijk {\em et al.} \cite{OdijkLekkerkerker}, which is a simplified version of the
%trial ODF used by Onsager \cite{Onsager}. For bidisperse systems of rods
%with different lengths, the Gaussian ODF successfully explained features
%like the fractionation effect, the widened biphasic gap \ \cite
%{OdijkLekkerkerker} and, somewhat later, the existence of triphasic and
%nematic-nematic equilibria \cite{LekkerVroeg}. A recent analysis by van Roij
%\cite{vanRoij96/2} based on elaborate numerical calculations of the exact
%high density ODF essentially confirmed all conclusions of  \cite
%{LekkerVroeg}, thus emphasizing the virtues of the Gaussian approximation.

As is clear from Eq. (\ref{0cluster}) the key ingredient in the Onsager theory is the excluded volume of two particles
which depends essentially on the shape of the particles under consideration.
In this thesis we shall model the particles as cylinders with length $L$ and $D$;
slender rods are then characterized by  $L/D\gg 1$ whereas thin platelets
have $L/D\ll 1$. Henceforth, we will use the
ratio of the largest to the shortest dimension of the particle, the {\em aspect ratio}, to quantify the anisometry of the particle.
The general expression for the excluded volume of
two different cylinders with lengths $L_{1}$ and $L_{2}$ and diameters $D_{1}$ and $D_{2}$ at mutual angle
$\gamma$ has been derived
in closed form by Onsager in a remarkable appendix to his paper \cite{onsager1949}. The result is
\begin{align}
v_{\text{excl}}(L_{1},D_{1};L_{2},D_{2};\gamma)=& \frac{\pi}{4}D_{1}D_{2}(D_{1}+D_{2})\left|\sin\gamma\right| +L_{1}L_{2}(D_{1}+D_{2})
\left|\sin\gamma\right| \nonumber  \\
&+L_{2}\left[\frac{\pi}{4}D_{2}^{2}+D_{1}D_{2}E(\sin\gamma)+\frac{\pi}{4}D_{1}^{2} \left|\cos\gamma \right| \right] \nonumber \\
&+L_{1}\left[\frac{\pi}{4}D_{1}^{2}+D_{1}D_{2}E(\sin\gamma)+\frac{\pi}{4}D_{2}^{2} \left|\cos\gamma \right| \right],
\label{0evonsager}
\end{align}
with $E(x)$ the complete elliptic integral of the second kind. For sufficiently anisometric particles characterized by a  large aspect ratio,
we may neglect the $\mathcal{O}(LD^{2})$-contributions arising from the particles' finite thicknesses\footnote{Strictly,
this is only justified if the orientational order in the nematic state is such that the  typical mutual angles $ \langle\langle\gamma \rangle\rangle $ are large compared
to $D/L$ (rods) or $L/D$ (platelets).} and  retain
only the leading order contribution, given by the first term (in case of thin platelets)
or the second one (for slender rods).
To assess the influence of  multi-particle correlations
Onsager gave some geometric arguments to estimate the following scaling behaviour of the triplet cluster integral
Eq. (\ref{0beta2}) for {\em isotropically oriented} thin rods
\begin{equation}
\frac{\beta_{2}}{\beta_{1}^{2}}\sim \mathcal{O}\left(\frac{D}{L}\ln\frac{L}{D}\right),
\end{equation}
which clearly vanishes for $L/D \rightarrow \infty $.
The decrease  has been  verified by means of Monte-Carlo simulations on hard spherocylinders by Frenkel
\cite{Frenkel87,Frenkel87err}
showing that higher order virial coefficients can be neglected only if $L/D\gg 100$.
The situation is much different for thin platelets for which Onsager estimated
\begin{equation}
\frac{\beta_{2}}{\beta_{1}^{2}}\sim \mathcal{O}(1), \label{0clusterplate}
\end{equation}
which is also true for spheres. This important result shows that the third and higher
virial terms cannot be neglected for thin platelets (not even in the limit $L/D\rightarrow 0$) \cite{Veerman}.

In the concentrated {\em nematic} state the interactions between the aligned particles are much
stronger due to steric hindering. For slender rods Onsager showed that the third virial coefficient in the aligned state
remains vanishingly small  only if the typical angle between the particles $\langle\langle \gamma \rangle\rangle $ is much larger than the so-called
{\em internal} angle $\gamma_{\text{int}}\sim D/L$ of the
rod.  If $\langle\langle \gamma \rangle\rangle$ is of the order $D/L$ the rods are nearly parallel
and the triplet cluster is always finite, as in Eq. (\ref{0clusterplate}), irrespective of $L/D$.
However, it turns out that the latter situation is not encountered for thin rods
since the ratio of the typical and internal angles $\langle\langle \gamma \rangle\rangle /\gamma_{\text{int}}$  can be shown to scale as $\sim L/D$, indicating
that higher virial terms vanish in the nematic phase  for $L/D\rightarrow \infty$ \cite{Vroege92}.
We can therefore conclude that  Onsager's  second virial theory for the isotropic-nematic transition
is an {\em exact} theory for rods in the limit $L/D \rightarrow \infty$ whereas  only qualitative results
can be expected for short rods (say $L/D\ll 100$) and  platelike particles.

\subsection{Mixtures}
In this thesis we shall be concerned with mixtures of anisometric particles
comprising either two distinctly different species (binary mixtures) or
a large number of particles with a continuously varying  size parameter
(polydisperse mixtures). Introducing  mole fractions $x_{j}=N_{j}/N$ of
species $j$,  the free energy of a mixture is given by a simple generalization of Eq. (\ref{0freetot}):
\begin{align}
\frac{\beta F}{N} &\sim \ln [\rho \mathcal{\bar{V}}]-1 + \sum_{j} x_{j} \ln x_{j} +
\sum_{j} x_{j} \int f_{j}(\Omega)\ln \left[ 4 \pi f_{j}(\Omega) \right] d \Omega \nonumber \\
&+\frac{\rho}{2}\sum_{j}\sum_{k}x_{j}x_{k} \iint  d \Omega d\Omega^{\prime}
f_{j}(\Omega)f_{k}(\Omega^{\prime})
v_{\text{excl}}^{jk}(\Omega,\Omega^{\prime}),  \label{0freetotmulti}
\end{align}
with $\mathcal{\bar{V}}=\prod_{j}\mathcal{V}_{j}^{x_{j}}$. The contribution following the ideal entropy is an {\em entropy of mixing} due to the fact that we are dealing with different species.
Although the free energy for mixtures is easily established,  the implications of Eq. (\ref{0freetotmulti})
are quite drastic. In particular, each species $j$ now has its own ODF
which must be normalized according to $\int f_{j}(d\Omega)d\Omega \equiv 1$. Formally minimizing the free energy
with respect to all ODFs then gives a {\em coupled} set of nonlinear equations:
\begin{equation}
\ln[4\pi f_{j}(\Omega)]=\lambda_{j} - \rho \sum_{k} x_{k} \int f_{k}(\Omega^{\prime})v_{\text{excl}}^{jk}(\Omega,\Omega^{\prime})
d\Omega^{\prime}, \label{0inteqmulti}
\end{equation}
which  is  progressively difficult to solve if the number of components increases.
A similar set of coupled equations, albeit not in integral form,
can be obtained from the Gaussian trial function approximation by inserting
$f_{G}(\alpha_{j};\theta)$ from Eq. (\ref{0ODF}) and minimizing with respect to all $\alpha_{j}$.
Moreover, in case of a phase coexistence between e.g. an isotropic ($I$) and a nematic ($N$) phase, the conditions for
mechanical and chemical equilibria require equal osmotic pressure $\Pi$ and chemical potentials $\mu_{j}$ for {\em all} species involved.
Hence, the coexistence equations are
\begin{align}
\Pi^{I}&=\Pi^{N} \nonumber \\
\mu^{I}_{j}&=\mu^{N}_{j} \qquad\qquad \text{for {\em all} $j$},
\end{align}
where we must realise that the composition $\{x_{j}\}$ may be different in each
phase due to fractionation effects.
These considerations indicate that the calculation of phase transition in mixtures is,
in general, a difficult task. For the binary mixtures to be considered
in Part I in this thesis, the equations are still manageable, in particular when
the Gaussian approximation is used.
However, for the polydisperse systems treated in Part II,  the situation is usually much worse so that
special numerical techniques have to be devised to solve
the coupled minimization equations along with the coexistence conditions,
as we shall see in Chapter 6.
}

\section[HPMC simulations of colloidal nematics]{Hard-particle Monte Carlo simulations of colloidal nematics}

\clearpage